\documentclass[12pt]{article}

\usepackage{EngReport}

\graphicspath{{Images/}}
\bibliography{Sources}
\onehalfspacing
\graphicspath{{images/}}
\geometry{letterpaper, portrait, includeheadfoot=true, hmargin=1in, vmargin=1in}

%\fontsize{font size}{vertsize (usually 1.2x)}\selectfont

\begin{document}
\renewcommand{\familydefault}{\rmdefault}

\begin{titlepage}
    \begin{center}
    {\fontsize{36}{42}\selectfont \bfseries ** Exploring Camunda **} 
    \\\vspace{20pt}
    {\LARGE ** Journey to learning Camunda **} \\
    \vspace{20pt}
    \textbf{** Amine Neggazi **}
    \vspace{8pt}
    \\ ** Prepared for, 23 / 07 / 2023 **
    \end{center}
\end{titlepage}

\pagestyle{fancy}
\fancyhf{}
\setlength{\headheight}{30pt}
\renewcommand{\headrulewidth}{0.4pt}
\renewcommand{\footrulewidth}{0.4pt}
\lhead{\large ** Exploring Camunda ** }
\rhead{\large ** \currentname ** }
\rfoot{\textbf{Page \thepage}}
\lfoot{}

\tableofcontents

\fontsize{12}{20}\selectfont{

\pagebreak

\section{Introduction}

In this article, I will share my journey of learning Camunda, an open-source platform for workflow and decision automation. As businesses strive for efficiency and agility, understanding tools like Camunda becomes crucial for streamlining processes and optimizing operations. I am very motivated to explore this powerful platform, and I embarked on a learning adventure to uncover the possibilities it offers.

Throughout this article, I will provide insights into the fundamentals of BPMN (Business Process Model and Notation) and its role in visualizing and modeling business processes. I will then dive into my experience with Camunda after using it in a small project, I'll also give a brief step by step learning journey from the initial installation and setup to exploring the Camunda Modeler, which allowed me to create clear and concise process diagrams.

I will also share valuable lessons learned,

  \subsection{Motivation and Goals}

I am motivated to learn camunda as it is a necessary tool for infrastructure automation and to also enhance productivity.
Here are some goals that I want to accomplish:

\begin{itemize}
  \item Gain a solid understanding of BPMN.
  \item Understand camunda's workflow engine.
  \item Effectively model and automate complex business processes.
  \item Camunda integration with other systems.
  \item Collaboration with co-workers.
\end{itemize}

\pagebreak

\section{What is BPMN ?}

  \subsection{Introduction}

Business Process Model and Notation was developed as a graphical notation to represent complex processes and address these challenges. It is maintained by the non-profit The Object Management Group (OMG) and employed by numerous organizations globally. The visual nature of BPMN enables greater collaboration between different teams.

  \subsection{Basics of BPMN}

Here are the foundational elements of BPMN:

  \begin{itemize}
    \item Process.
    \item Activities (Tasks, Sub-Processes, and Call activities).
    \item Events (Start Events, Intermediate Events, and End Events).
    \item Gateways (Exclusive, Parallel, and Inclusive Gateways)
    \item Sequence Flow.
    \item Data Objects.
  \end{itemize}

  \subsection{Learning Process and Resources}
I started by learning the concepts of each elements used in BPMNs (tasks, symbols, activities...), and then since I had previous knowledge of modeling languages and UML diagrams, so it all seemed similiar.
Then, I created some simple examples :
One of them was automating a Car Starting System (Each car executes some checks before starting).
I also read some best practices on using BPMNs.

\pagebreak

\section{What is Camunda ?}
  \subsection{Introduction}
  \subsection{Installation and Setup}

\pagebreak

\section{Creating a Simple Process}
  \subsection{Designing a Process using Camunda Modeler}
  \subsection{Executing the Process with Camunda's Workflow Engine}

\pagebreak

\section{Advanced Topics}
  \subsection{User Task Management}
  \subsection{Automation using DMN}
  \subsection{Process Optimization}

\pagebreak

\section{What I Learned}
  \subsection{Challenges, Successes, and Some Takeaways}

\pagebreak

\section{Conclusion}

}

\printbibliography

\end{document}
